\addcontentsline{toc}{chapter}{TÓM TẮT}
\chapter*{Tóm tắt}

Ngày này có rất nhiều hệ thống được xây dựng dựa trên các mô hình tính toán khác nhau.
Với mỗi mô hình tính toán lại có các ưu, nhược điểm khác nhau.
Để đánh giá mỗi mô hình tính toán, cần triển khai mỗi mô hình trên hệ thống trong một thời gian dài.
Sau khi triển khai xong, các doanh nghiệp tổ chức rất khó để so sánh giữa các mô hình tính toán khác nhau.
Bởi vì mỗi mô hình được chạy trong các khoảng thời gian khác nhau và có tập người sử dụng khác nhau
trong các thời điểm.

Do đó các doanh nghiệp nảy sinh ra nhu cầu cần có một hệ thống toàn diện để giải quyết bài toán trên.
Hệ thống này có thể đánh giá các mô hình tính toán khác nhau một cách chính xác và công bằng.
Từ đó doanh nghiệp có thể biết được mô hình tính toán nào tốt hơn mà không cần phải tốn quá nhiều
thời gian thử nghiệm trên từng mô hình.

Hệ thống A/B Testing được ra đời để hoàn thành mục đích đó. Khi sử dụng hệ thống, bất cứ người dùng
nào cũng có thể khởi tạo các mô hình với các thông số khác nhau để so sánh. Hệ thống A/B Testing cũng
có thể tích hợp với các hệ thống khác để chạy thử nghiệm, từ đó thu thập dữ liệu để đánh giá các mô
hình.

\textbf{Từ khoá}: A/B Testing, thử nghiệm, đánh giá.
